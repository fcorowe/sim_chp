\RequirePackage[l2tabu,orthodox]{nag}
\documentclass[11pt,letterpaper]{article}
\usepackage[T1]{fontenc}
\usepackage[utf8]{inputenc}
\usepackage{crimson}
\usepackage{helvet}
\usepackage[strict,autostyle]{csquotes}
\usepackage[USenglish]{babel}
\usepackage{microtype}
\usepackage{authblk}
\usepackage{booktabs}
\usepackage{caption}
\usepackage{endnotes}
\usepackage{geometry}
\usepackage{graphicx}
\usepackage{hyperref}
\usepackage{natbib}
\usepackage{rotating}
\usepackage{setspace}
\usepackage{titlesec}
\usepackage{url}
\usepackage{soul}
\usepackage[dvipsnames]{xcolor}
\usepackage[many]{tcolorbox}
\newtcolorbox{mybox}{colback = black!5!gray!50, colframe = black!75!black, segmentation style={solid}} % create text box
\usepackage{hanging}

% location of figure files, via graphicx package
\graphicspath{{./figures/}}

% configure the page layout, via geometry package
\geometry{
	paper=letterpaper,
	top=4cm,
	bottom=4cm,
	left=4cm,
	right=4cm}
\setstretch{1.02}
\clubpenalty=10000
\widowpenalty=10000

% set section/subsection headings as the sans serif font
\titleformat{\section}{\normalfont\sffamily\large\bfseries}{\thesection.}{0.3em}{}
\titleformat{\subsection}{\normalfont\sffamily\small\bfseries}{\thesubsection.}{0.3em}{}

% make figure/table captions sans-serif small font
\captionsetup{font={footnotesize,sf},labelfont=bf,labelsep=period}

% configure pdf metadata and link handling
\hypersetup{
	pdfauthor={Francisco Rowe, Robin Lovelace, Adam Dennett},
	pdftitle={Spatial Interaction Modelling: A Manisfesto},
	pdfsubject={Spatial Interaction Modelling: A Manisfesto},
	pdfkeywords={Spatial interaction modelling, Gravity modelling, Machine learning, Geographic data science},
	pdffitwindow=true,
	breaklinks=true,
	colorlinks=false,
	pdfborder={0 0 0}}

\title{Spatial Interaction Modelling: A Manisfesto\footnote{\textbf{Citation}: Rowe, F. Lovelace, R. Dennett, A., 2022. Spatial Interaction Modelling: A Manisfesto. In: Wolf, L., Heppenstall, A., and Harris, R. (eds) Edward Elgar Books}}
\author[1]{Francisco Rowe \thanks{\textit{Email}: F.Rowe-Gonzalez@liverpool.ac.uk}}
\affil[1]{Geographic Data Science Lab, Department of Geography and Planning, University of Liverpool, Liverpool, United Kingdom}
\author[2]{Robin Lovelace \thanks{\textit{Email}: R.Lovelace@leeds.ac.uk}}
\affil[2]{Institute for Transport Studies, University of Leeds, Leeds, United Kingdom}
\author[3]{Adam Dennett \thanks{\textit{Email}: a.dennet@ucl.ac.uk}}
\affil[3]{The Bartlett Centre for Advanced Spatial Analytics, University College London, London, United Kingdom}

\date{}

% From pandoc:
% https://github.com/jgm/pandoc-templates/blob/master/default.latex
\newlength{\cslhangindent}
\setlength{\cslhangindent}{1.5em}
\newlength{\csllabelwidth}
\setlength{\csllabelwidth}{3em}
\newlength{\cslentryspacingunit} % times entry-spacing
\setlength{\cslentryspacingunit}{\parskip}
\newenvironment{CSLReferences}[2] % #1 hanging-ident, #2 entry spacing
 {% don't indent paragraphs
  \setlength{\parindent}{0pt}
  % turn on hanging indent if param 1 is 1
  \ifodd #1
  \let\oldpar\par
  \def\par{\hangindent=\cslhangindent\oldpar}
  \fi
  % set entry spacing
  \setlength{\parskip}{#2\cslentryspacingunit}
 }%
 {}
\usepackage{calc}
\newcommand{\CSLBlock}[1]{#1\hfill\break}
\newcommand{\CSLLeftMargin}[1]{\parbox[t]{\csllabelwidth}{#1}}
\newcommand{\CSLRightInline}[1]{\parbox[t]{\linewidth - \csllabelwidth}{#1}\break}
\newcommand{\CSLIndent}[1]{\hspace{\cslhangindent}#1}
\setlength{\emergencystretch}{3em} % prevent overfull lines
\providecommand{\tightlist}{%
  \setlength{\itemsep}{0pt}\setlength{\parskip}{0pt}}

% https://stackoverflow.com/questions/41052687/rstudio-pdf-knit-fails-with-environment-shaded-undefined-error
\usepackage{color}
\usepackage{fancyvrb}
\newcommand{\VerbBar}{|}
\newcommand{\VERB}{\Verb[commandchars=\\\{\}]}
\DefineVerbatimEnvironment{Highlighting}{Verbatim}{commandchars=\\\{\}}
% Add ',fontsize=\small' for more characters per line
\usepackage{framed}
\definecolor{shadecolor}{RGB}{248,248,248}
\newenvironment{Shaded}{\begin{snugshade}}{\end{snugshade}}
\newcommand{\AlertTok}[1]{\textcolor[rgb]{0.94,0.16,0.16}{#1}}
\newcommand{\AnnotationTok}[1]{\textcolor[rgb]{0.56,0.35,0.01}{\textbf{\textit{#1}}}}
\newcommand{\AttributeTok}[1]{\textcolor[rgb]{0.77,0.63,0.00}{#1}}
\newcommand{\BaseNTok}[1]{\textcolor[rgb]{0.00,0.00,0.81}{#1}}
\newcommand{\BuiltInTok}[1]{#1}
\newcommand{\CharTok}[1]{\textcolor[rgb]{0.31,0.60,0.02}{#1}}
\newcommand{\CommentTok}[1]{\textcolor[rgb]{0.56,0.35,0.01}{\textit{#1}}}
\newcommand{\CommentVarTok}[1]{\textcolor[rgb]{0.56,0.35,0.01}{\textbf{\textit{#1}}}}
\newcommand{\ConstantTok}[1]{\textcolor[rgb]{0.00,0.00,0.00}{#1}}
\newcommand{\ControlFlowTok}[1]{\textcolor[rgb]{0.13,0.29,0.53}{\textbf{#1}}}
\newcommand{\DataTypeTok}[1]{\textcolor[rgb]{0.13,0.29,0.53}{#1}}
\newcommand{\DecValTok}[1]{\textcolor[rgb]{0.00,0.00,0.81}{#1}}
\newcommand{\DocumentationTok}[1]{\textcolor[rgb]{0.56,0.35,0.01}{\textbf{\textit{#1}}}}
\newcommand{\ErrorTok}[1]{\textcolor[rgb]{0.64,0.00,0.00}{\textbf{#1}}}
\newcommand{\ExtensionTok}[1]{#1}
\newcommand{\FloatTok}[1]{\textcolor[rgb]{0.00,0.00,0.81}{#1}}
\newcommand{\FunctionTok}[1]{\textcolor[rgb]{0.00,0.00,0.00}{#1}}
\newcommand{\ImportTok}[1]{#1}
\newcommand{\InformationTok}[1]{\textcolor[rgb]{0.56,0.35,0.01}{\textbf{\textit{#1}}}}
\newcommand{\KeywordTok}[1]{\textcolor[rgb]{0.13,0.29,0.53}{\textbf{#1}}}
\newcommand{\NormalTok}[1]{#1}
\newcommand{\OperatorTok}[1]{\textcolor[rgb]{0.81,0.36,0.00}{\textbf{#1}}}
\newcommand{\OtherTok}[1]{\textcolor[rgb]{0.56,0.35,0.01}{#1}}
\newcommand{\PreprocessorTok}[1]{\textcolor[rgb]{0.56,0.35,0.01}{\textit{#1}}}
\newcommand{\RegionMarkerTok}[1]{#1}
\newcommand{\SpecialCharTok}[1]{\textcolor[rgb]{0.00,0.00,0.00}{#1}}
\newcommand{\SpecialStringTok}[1]{\textcolor[rgb]{0.31,0.60,0.02}{#1}}
\newcommand{\StringTok}[1]{\textcolor[rgb]{0.31,0.60,0.02}{#1}}
\newcommand{\VariableTok}[1]{\textcolor[rgb]{0.00,0.00,0.00}{#1}}
\newcommand{\VerbatimStringTok}[1]{\textcolor[rgb]{0.31,0.60,0.02}{#1}}
\newcommand{\WarningTok}[1]{\textcolor[rgb]{0.56,0.35,0.01}{\textbf{\textit{#1}}}}


\begin{document}

\maketitle


\textbf{NOTE} This is a preprint version of the chapter. Please do not redistribute without
permission of the authors. You can get in touch with Francisco Rowe at
F.Rowe-Gonzalez@liverpool.ac.uk; Robin Lovelace at R.Lovelace@leeds.ac.uk or Adam Dennett a.dennet@ucl.ac.uk

\begin{abstract}


%\vspace{1cm}
\end{abstract}



\pagebreak

\hypertarget{definition-and-uses}{%
\section{Definition and Uses}\label{definition-and-uses}}

For over 70 years, spatial interaction models (SIMs) have been the workmodel to understand spatial interactions between entities at different locations in physical space.
SIMs represent mathematical formulations through which the spatial interaction between geographic places encoded in flows of people, information and goods can be rendered.
Intuitively, these models seek to represent the spatial interaction between places as a function of three components: origin characteristics, destination characteristics and the separation between origins and destinations.
Originally adapted from physics, spatial flows between an origin and a destination were concieved to be proportional to their gravitational force and inversely related to their spatial separation.
Characteristics of origins and destinations are used to represent gravitational forces pushing and/or pulling people, information and goods from and to specific locations, and different forms of distance and costs are used to represent the deterring effects of geographical separation on spatial flows.

SIMs have been instrumental and applied in a wide range of contexts to support data analysis and decision making in retail, transport, housing, epidemiology, public health, land use, urban and population modelling and planning projection and forecasting contexts.
SIMs are generally used for two key purposes.
A key purpose is the \emph{prediction} of the size and direction of spatial flows.
SIMs have been widely used to predict the impact of the development of new service units, such as shopping stores, healthcare facilities and housing units on the potential demand for associated services and traffic patterns.
Predictions from such analyses enable the identification of optimal locations and size for potential new service units.
A second key purpose of SIMs is \emph{inference} about the factors shaping the spatial interactions in a network of flows.
SIMs have been used to determine and understand the influence of retail store on consumers' store choices and place attributes on migration decisions and commuting patterns.
SIMs have also been used to delineate geographical areas of service and retail catchment areas.

Formally SIMs take different forms.
Newtonian gravity models are probably the most widely known and used form of spatial interaction models in social sciences.
Adapted from physics, the basic gravity model assumes that the interactions \(T_{i j}\) between an origin \(i\) and a destination \(j\) in the form of flows can be understood as a function of driving forces like masses \(V_{i}\) and \(W_{j}\) , and a measure of spatial separation \(c_{ij}\) .
Areas are assumed to interact in a positively reinforcing way that is multiplicative of their masses, and at the same time their interactions are expected to diminish with the intervening role of spatial separation.
Spatial separation is generally measured by distance, cost or time involved in the interaction, and is often represented by a distance-decay function.
The model also requires a \(k\) balacing factor between expected and observed flows, and \(\beta\) parameter representing the deterring effect of spatial separation, or distance.
The key task in a gravity model is to estimate these parameters (i.e.~\(k\) and \(\beta\) ).
A typical notation for the model is:

\[
T_{i j}=k \frac{V_{i} W_{j}}{c_{i j}^{\beta}}
\]

In practice, a matrix of flows, between a set of origins, a set of destinations and a measure of spatial separation between origins and destinations, is the key input for SIMs.
A family of SIMs encapsulating four distinctive shapes is typically considered (Wilson 1971).
A \emph{unconstrained} formulation of the model to ensure that the total sum of the predicted flows from a gravity model be equal the total sum of the observed flows across all origins and destinations.
\emph{Constrained} versions are used to ensure that specific restrictions are met.
Three general formulations of constrained models are used: production-constrained, attraction-constrained and doubly constrained models.
\emph{Production-constrained} forms are used to constrain a model so that the predicted number of trips emanating from each origin is equal to the observed number of trips.
\emph{Attraction-constrained} forms are used to constrain a model so that the predicted number of trips terminating at each destination is equal to the observed number of trips.
\emph{Doubly constrained} models combine these two sets of constraints.

SIMs have been extended in six key ways.
First, social science theory has been infused to underpin and enrich SIMs.
SIMs were originally conceptualised as a mathematical formulation to represent observed relationships between origins and destinations encoded in a origin-destination matrix, and generate accurate prediction of spatial flows.
Field-specific theories have been used to extend and develop the fundamental structure of SIMs to understand retail, trade, transport, communication, migration and mobility patterns.
Second, more sophisticated measures to capture the influence of origins and destinations have been developed.
The specification of SIMs has moved away from relying on population size to approximate the propulsive effect of origins and attractive force of destinations, to capture local economic, political, cultural and social differences across origins and destination, and recognise propulsive and attractive effects at play in both origins and destinations.
Third, measures of spatial separation have also been sophisticated to more appropriately reflect the geographical, physiological and financial distance and costs between origins and destinations (Schwartz 1973), as well as the system of road networks {[}{]} and the spatial distribution of human settlements (Niedomysl et al. 2017).
Fourth, considerable methodological work has been done to conceptualise and operationalise the influence of spatial structure in SIMs (Oshan 2021).
Five generalisable approaches have been proposed to account for spatial structure in distinctive ways: the competing destination model by including an accessibility measure (Fotheringham 1983); the Box--Cox transform by using a Box--Cox functional form of distance {[}{]}; spatial choice modelling by accounting for destination alternative substitution patterns (Hunt, Boots, and Kanaroglou 2004); spatial econometric modelling {[}{]}; and eigenvector spatial filtering by incorporating spatial dependence or spatial autocorrelation {[}{]}.
Fifth, frameworks to calibrate SIMs have evolved.
Early methods of calibration comprised linear programming and non-linear optimization.
Regression methods have now become prominent evolving from linear log-normal model formulations using ordinary least squares (OLS) and maximum likelihood estimation procedures, through generalised linear frameworks using iteratively reweighted least squares (IWLS), especially Poisson and Negative Binomial distributions {[}{]}, to more sophisticated multilevel and discrete choice models to capture origin-, destination- or origin-destination-specific parameters {[}{]}.
Sixth, alternative frameworks of spatial interaction modelling have been developed to incorporate different ways of conceptualising the decision-making process of choosing a destination location.
The radiation (Simini et al. 2012), exploration and preferential return (Pappalardo et al. 2015) and random utility models (McFadden 1974) have become prominent frameworks to model spatial flows, particularly human mobility flows.

Thus, considerable progress has been made on advancing the theoretical and methodological underpinnings of SIMs.
Yet, major challenges remain in terms of reproducibility, calibration and \emph{Big Data} modelling.
We believe that addressing these challenges are key to facilitate the application of SIMs, extend existing modelling approaches, leverage the greater geographical and temporal breadth, depth, scale and timeliness of \emph{Big Data} and ultimately enhance our understanding human interactions and our social world.

\hypertarget{challenges}{%
\section{Challenges}\label{challenges}}

In this section, we want to describe why the areas identified are seen as challenges and in which ways.

\hypertarget{reproducibility}{%
\subsection{Reproducibility}\label{reproducibility}}

Reproducibility of SIMs has not been a prominent area of attention in past research.
Replicating most SIM application would certainly be a challenge.
Such situation is unfortunate as it is likely to have limited the applicability and portability of sophisticated frameworks to estimate SIMs and capture the effects of spatial structure on spatial flows.
In fact many sophisticated SIMs have been developed to capture the effects of spatial structure on spatial flows, but the accessibility and applicability of such models remained limited.
Arguably even the application of basic formulations of SIMs tends to represent a daunting challenge for beginning modelers of spatial flow data.
The lack of reproducibility as a standard research practice has tended to generate ``black box'' data analyses, preventing the validation, verification and comparability of SIM estimates.

Understandably, reproducibility was not concern during the 1970s and 1980s, when significant progress was made on developing SIMs.
Computer hardware, software and know-how needed to develop and deploy SIMs were unavailable to most people.
Even when consumer laptops became widely available and more affordable during the 2000s, few well-known user-friendly off-the-shelf options to implement SIMs existed.
Yet, today, computer software and hardware are highly affordable and learning computing programming has become increasingly accessible.
With the advent of open science, resources to learn computer programming in popular data science languages, such as \emph{R} and \emph{Python} have become widely available.
Nevertheless, most SIM applications cannot still be reproduced.

Yet, reproducibility of SIM estimates can yield key benefits for future research, impact and training .
Reproducible research can boots citations and facilitate comparability of research results enabling the application of existing methods to a wide range of contexts (Brunsdon 2016). Reproducibility can also enhance accountability, validation and verification of research findings by increasing transparency Brunsdon (2016).
By transparently sharing open data products, including reproducible code and input data (Arribas-Bel et al. 2021), used in the analysis, results can be validated and lessons can be learnt from the limitations and strengths of the application of particular methods.
Reproducible open products can also increase the portability of existing work.
New research can build on existing code and data focusing on addressing new novel questions, and avoiding reinvention of wheels and associated waste of time.
Reproducible work can facilitate the generation of updates as new data become available.
Reproducible open data products can also be used as communication and impact strategy expanding the original purpose of research findings (Nüst and Pebesma 2021).
Reproducible code and data can be used to provide educational training and enable practitioners to address policy questions which were not outside the scope of the original research project (Rowe et al. 2020).

\hypertarget{calibration}{%
\subsection{Calibration}\label{calibration}}

Calibration is at the centre of what makes spatial interaction models useful to researchers and practitioners.
As Openshaw (1975) describes it, ``calibration is the process of providing estimates of the unknown parameters we have identified as the independent variables of the model.'' In a basic gravity SIM formulation, we seek to estimate three parameters - as identified above - relating to origin mass, destination mass and a parameter determining the frictional effect of spatial separation between origins and destinations.

Yet, we identified key challenges relating to the calibration of SIMs.
Calibration requires both data and software.
Forty or fifty years ago when many of the theoretical foundations of spatial interaction modelling were laid, the data landscape was somewhat different from today -- spatial flow data were rarely available in volume and certainly not at the sort of temporal rhythm they are now where, for example, supermarket loyalty-card holders generate origin/destination revenue flow data from residential origins to store destinations at daily frequencies over time periods than can cover many years or even decades.
As such, we might be forgiven for expecting that even if the science and theory underpinning the models has not developed very much, the software and processes facilitating calibration -- relating empirical observations to the theoretical representations embodied in the models -- might have.
But in many ways, they have not.
Today spatial flow data are ubiquitous.
They can be derived from digital traces collected across various sensor networks involving mobile phones, social media, loyalty cards, smart card tickets and credit cards.
Accessing interaction data is thus not the barrier it once was.

Yet, making sense of these data -- understanding where commuting flows are unexpected, or differences in customer profiles shopping at similar retail stores -- remains a challenge.
Much of the challenge is because there is a dearth of knowledge within geographical education.
Despite SIMs underpinning many social and geographical processes, these models are not taught to undergraduate geography students in the same way as, say, regression models are taught to economists or social psychologists.
As such, it is not immediately obvious, even for those with geography degrees, how anyone might go about fitting their spatial flow data to a theoretical model, calibrating the parameters and revealing properties of their system.
But, why have cohorts of undergraduate geography students not been taught how to fit these models and explore systems of spatial interaction?

Accessibility seems to represent an obstacle for the wider applicability of SIMs.
We discussed issues around reproducibility challenging the practical application of SIMs, but that with packages like \texttt{SPINT} (*Oshan reference) and our own efforts with the \texttt{simodels} package in \emph{R} (see Section \ref{enabling-infrastructure}, the tide is beginning to turn.
Yet, calibration remains a challenging sub-topic for reproducibility and wider accessibility of these models, particularly where algorithms which are able to calibrate parameters have remained locked away - either behind dense algebraic notation in dusty papers from the 1970s, or where they have found their way into software behind paywalls.

Ironically, effective calibration routines have been available for as long as students have been running regression models in their introductory statistics classes.
Occasional references can be found in the historic literature (e.g.~Fotheringham and Webber (1980); Flowerdew and Aitken (1982)) which lift the curtain and reveal that through reformulating the classic Wilsonian entropy maximising spatial interaction model as either a logged OLS regression model or a GLM utilising a Poisson or negative binomial distributions, multiple parameters can be calibrated easily.
These models are all available in common statistical software packages; but for most trying to make sense of the field coming across papers by some of the doyens (for it was and still is a male-dominated field) of the scene, while undoubtedly mathematically and theoretically rigorous, notes on calibration were at best reduced to a passing reference to `least squares' or at worst a lengthy derivation of maximum likelihood of Newton Raphson methods.

\hypertarget{big-data-modelling}{%
\subsection{\texorpdfstring{Big Data modelling }{Big Data modelling }}\label{big-data-modelling}}

As argued above, new technologies have enabled the emergence of `Big Data' through the production and storage of large volumes of digital data.
Much of these data contain location information and hence offer an opportunity to derive spatial flow data and understand spatial interactions between places.
Big Data offer unique opportunities to study spatial interactions at unprecedented detailed geographical and temporal scales in real or near real-time across extensive populations and geographical areas.
However, leveraging on these opportunities also involves major challenges for the analysis and modelling of spatial interactions (Rowe 2021).

Limited research has focused on explicitly capturing patterns of non-stationarity in spatial interaction modelling despite the availability of suitable methods (Oshan 2021).
While the issues of calibration discussed in Section \ref{calibration} may have represented a barrier, lack of large enough data sets may have also hindered progress.
Before the emergence of Big Data, the most common form of spatial flow data were cross-sections of origin-destination matrices derived from censuses or surveys, offering information at coarse geographical scales and population subgroups.
The rise of data now offers more granular and denser volumes of data to capture patterns of spatial, temporal and population non-stationarity.
The adoption and adaption of more sophisticated modelling frameworks will thus be required to effectively model these patterns.

Big Data also represent challenges for statistical inference.
An standard practice in social sciences is to be guided by \emph{P} values.
Regression model estimates with \emph{P} values below 0.05 threshold are commonly considered statistically significant and thus these estimates take central stage in most analyses.
Modelling estimates derived from large data sets, however, often render \emph{P} values below this threshold, calling for the adoption of alternative approaches.
Such situation aligns with more general calls for a stop to the use of P values in the conventional, dichotomous way - to decide whether a result supports or refutes a scientific hypothesis.
Additionally, Big Data offer an opportunity to embrace causal inference approaches.
Reliance on cross-sectional data hindered the wide adoption of causal inference approaches to study the determinants of spatial interactions.
Big Data now provides large, detailed longitudinal data sets to track spatial interactions and establish cause-and-effect relationships.
Obtaining suitable time-varying data on relevant factors believed to shape the dynamics of spatial interactions may remain an obstacle.
While data may be available, spatial integration of data may not be possible because of ethical considerations and data governance issues.

New methods are also required to handle, analyse and store large data sets.
Traditional SIM frameworks were designed to identify significant relationships in small sample sizes with known properties.
Big data are not collected for research purposes.
They are an unintended consequence of administrative processes or social interactions and need to be reengineered for research.
Handling Big Data requires a wider and new digital skills set, largely based on machine learning, artificial intelligence and coding, in addition to greater knowledge of computing technology (e.g.~Jupyter notebooks, Github and Docker) as well as scalability and parallelisation approaches.
Except for a few centres, current university geography programmes and infrastructures are largely unprepared to deliver the required training.
A multidisciplinary approach is needed to integrate computational training into human geography.
Machine learning and artificial intelligence are likely to enhance prediction outcomes generated from SIMs.
The idea of using machine learning and artificial intelligence to model spatial interactions is not new, but the application has been limited due to the lack of large data sets.
Machine learning and artificial intelligence are data hungry, requiring millions of data for effective training and validation.
The rise of data provide an opportunity to promote the wider adoption of these models.

\hypertarget{the-way-forward}{%
\section{The Way Forward}\label{the-way-forward}}

This section will describe the areas which should be developed to address the identified challenges.

\hypertarget{enabling-infrastructure}{%
\subsection{\texorpdfstring{Enabling Infrastructure }{Enabling Infrastructure }}\label{enabling-infrastructure}}

Developing the essential infrastructure is key to enhance the reproducibility and facilitate the calibration of SIMs.
Software, open science and digital technology (i.e.~computational notebooks and Docker) are important elements to develop an ecosystem that fosters reproducible SIMs, provides adequate training and facilitate the application of SIMs.
We believe that an essential building block in this ecosystem is user-friendly, efficient, open source software.
Partly motivated by this chapter, we have developed the \texttt{simodels}\footnote{short for spatial interaction models} \emph{R} package.
\footnote{the primary motivation was the need to develop SIMs to represent trips for non-commuting or non-school purposes in Ireland as part of a contract with Transport Infrastructure Ireland.} Below we present a reproducible example
. \texttt{simodels} enables to develop SIMs taking geographic data sets as an input in a few lines of code
. To install the package, run
:

\begin{Shaded}
\begin{Highlighting}[]
\FunctionTok{install.packages}\NormalTok{(}\StringTok{"simodels"}\NormalTok{)}
\end{Highlighting}
\end{Shaded}

\texttt{simodels} does not just provide functions for running and fitting (finding parameters to minimise model-observation differences).
It provides a framework for developing SIMs and creating new functions implementing different types of SIM and using a variety of pre-existing modelling tools in SIMs.
We will also install \texttt{tidyverse} for intuitive data processing functionality and load the packages:

\begin{Shaded}
\begin{Highlighting}[]
\FunctionTok{install.packages}\NormalTok{(}\StringTok{"tidyverse"}\NormalTok{)}
\end{Highlighting}
\end{Shaded}

\begin{Shaded}
\begin{Highlighting}[]
\FunctionTok{library}\NormalTok{(simodels)}
\FunctionTok{library}\NormalTok{(tidyverse)}
\end{Highlighting}
\end{Shaded}

The starting `point' is geographic entities representing trip start, end (for `multi-partite' models) or intermediate points.
We use the word `features' because almost all SIMs use input data sets that are compliant with the `simple features' open specification ((OGC) Open Geospatial Consortium Inc 2011), typically imported from files encoded in proprietary the Shapefile (\texttt{.shp}) or open GeoPackage (\texttt{.gpkg}), GeoJSON (\texttt{.geojson}) or other geographic file formats.
\emph{R} has a mature ecosystem for working with geographic file formats, so we can use the \texttt{sf} package:

\begin{Shaded}
\begin{Highlighting}[]
\NormalTok{u\_origins }\OtherTok{=} \StringTok{"origin\_zones.geojson"}
\NormalTok{f\_origins }\OtherTok{=} \FunctionTok{basename}\NormalTok{(u\_origins)}
\NormalTok{u\_destinations }\OtherTok{=} \StringTok{"destination\_points.geojson"}
\NormalTok{f\_destinations }\OtherTok{=} \FunctionTok{basename}\NormalTok{(u\_destinations)}
\end{Highlighting}
\end{Shaded}

\begin{Shaded}
\begin{Highlighting}[]
\FunctionTok{download.file}\NormalTok{(u\_origins, }\AttributeTok{destfile =}\NormalTok{ f\_origins)}
\FunctionTok{download.file}\NormalTok{(u\_destinations, }\AttributeTok{destfile =}\NormalTok{ f\_destinations)}
\end{Highlighting}
\end{Shaded}

\begin{Shaded}
\begin{Highlighting}[]
\NormalTok{origin\_zones }\OtherTok{=}\NormalTok{ sf}\SpecialCharTok{::}\FunctionTok{read\_sf}\NormalTok{(}\StringTok{"origin\_zones.geojson"}\NormalTok{)}
\NormalTok{destination\_points }\OtherTok{=}\NormalTok{ sf}\SpecialCharTok{::}\FunctionTok{read\_sf}\NormalTok{(}\StringTok{"destination\_points.geojson"}\NormalTok{)}
\end{Highlighting}
\end{Shaded}

The code chunk above demonstrates importing specific data objects : 1) a simple features object with `multipolygon' geometries representing administrative zones that constitute trip origins; and, 2) a simple features object with `point' geometries representing two popular pubs in Leeds as trip destinations.
Before creating SIMs representing travel to these two pubs in Leeds, we first perform some exploratory data analysis (EDA) to illustrate the input data.

\begin{Shaded}
\begin{Highlighting}[]
\NormalTok{origin\_zones }\SpecialCharTok{\%\textgreater{}\%} 
  \FunctionTok{ggplot}\NormalTok{() }\SpecialCharTok{+}
  \FunctionTok{geom\_histogram}\NormalTok{(}\FunctionTok{aes}\NormalTok{(}\AttributeTok{x =}\NormalTok{ to\_pubs), }\AttributeTok{binwidth =} \DecValTok{10}\NormalTok{)}
\NormalTok{origin\_zones }\SpecialCharTok{\%\textgreater{}\%} 
  \FunctionTok{ggplot}\NormalTok{() }\SpecialCharTok{+}
  \FunctionTok{geom\_sf}\NormalTok{(}\FunctionTok{aes}\NormalTok{(}\AttributeTok{fill =}\NormalTok{ to\_pubs), }\AttributeTok{alpha =} \FloatTok{0.5}\NormalTok{) }\SpecialCharTok{+}
  \FunctionTok{geom\_sf}\NormalTok{(}\AttributeTok{data =}\NormalTok{ destination\_points)}
\end{Highlighting}
\end{Shaded}

\includegraphics[width=0.5\linewidth]{main_files/figure-latex/unnamed-chunk-7-1} \includegraphics[width=0.5\linewidth]{main_files/figure-latex/unnamed-chunk-7-2}

For many applications, the most important function in the \texttt{simodels} package is \texttt{si\_to\_od()}:

\begin{Shaded}
\begin{Highlighting}[]
\NormalTok{od\_zones\_to\_points }\OtherTok{=} \FunctionTok{si\_to\_od}\NormalTok{(origin\_zones, destination\_points)}
\FunctionTok{class}\NormalTok{(od\_zones\_to\_points)}
\FunctionTok{nrow}\NormalTok{(od\_zones\_to\_points)}
\FunctionTok{names}\NormalTok{(od\_zones\_to\_points)}
\end{Highlighting}
\end{Shaded}

As shown in the output above, the result is a data frame with 94 rows, representing the full combination of trips from each of the 47 origin zones to each of the 2 destinations.
The names in the data frame refer to variables for origin and destination locations.
When working on large input data sets, the `full matrix' of combinations can get unhelpfully large: an OD dataset from every MSOA to every pub in England, for example, would results in a data set with 350,000,000 (350 million) rows.
To reduce data set sizes, a `sparse matrix' representing only OD pairs below a certain distance threshold can be created by adding a \texttt{max\_dist} argument as shown below.
The maximum Euclidean distance between zone centroids and point destinations needs to be set, at 5 km in this example.
Note `si\_to\_od` will expect Euclidean distance to be provided in the origin-destination data set.

\begin{Shaded}
\begin{Highlighting}[]
\NormalTok{od\_zones\_to\_points }\OtherTok{=} \FunctionTok{si\_to\_od}\NormalTok{(origin\_zones, }
\NormalTok{                              destination\_points, }
                              \AttributeTok{max\_dist =} \DecValTok{5000}\NormalTok{)}
\end{Highlighting}
\end{Shaded}

The resulting origin-destination data set is smaller (79 rows compared with 94 rows previously).
While in the existing example this does lead to a major reduction, this process can greatly speed-up SIM processing, modelling and visualisation run times dealing with large data sets.

We can now specify a simple SIM model as follows:

\begin{Shaded}
\begin{Highlighting}[]
\NormalTok{gravity\_model }\OtherTok{=} \ControlFlowTok{function}\NormalTok{(beta, d, m, n) \{}
\NormalTok{  m }\SpecialCharTok{*}\NormalTok{ n }\SpecialCharTok{*} \FunctionTok{exp}\NormalTok{(}\SpecialCharTok{{-}}\NormalTok{beta }\SpecialCharTok{*}\NormalTok{ d }\SpecialCharTok{/} \DecValTok{1000}\NormalTok{)}
\NormalTok{\} }
\end{Highlighting}
\end{Shaded}

and implement it with the following command:

\begin{Shaded}
\begin{Highlighting}[]
\NormalTok{od\_to\_pubs\_result }\OtherTok{=}\NormalTok{ od\_zones\_to\_points }\SpecialCharTok{\%\textgreater{}\%} 
  \FunctionTok{si\_calculate}\NormalTok{(}\AttributeTok{fun =}\NormalTok{ gravity\_model, }
               \AttributeTok{m =}\NormalTok{ origin\_to\_pubs,}
               \AttributeTok{n =}\NormalTok{ destination\_size,}
               \AttributeTok{d =}\NormalTok{ distance\_euclidean,}
               \AttributeTok{beta =} \FloatTok{0.5}\NormalTok{,}
               \AttributeTok{constraint\_production =}\NormalTok{ origin\_to\_pubs)}
\end{Highlighting}
\end{Shaded}

We can check the results:

\begin{Shaded}
\begin{Highlighting}[]
\FunctionTok{sum}\NormalTok{(od\_to\_pubs\_result}\SpecialCharTok{$}\NormalTok{interaction)}
\end{Highlighting}
\end{Shaded}

\begin{verbatim}
## [1] 2903
\end{verbatim}

\begin{Shaded}
\begin{Highlighting}[]
\FunctionTok{sum}\NormalTok{(origin\_zones}\SpecialCharTok{$}\NormalTok{to\_pubs)}
\end{Highlighting}
\end{Shaded}

\begin{verbatim}
## [1] 2903
\end{verbatim}

As shown above, the total number of trips is the same in the OD data as in the zone level data.
We can visualise the result as follows, resulting in Figure \ref{fig:pubresmap}:

\begin{Shaded}
\begin{Highlighting}[]
\FunctionTok{library}\NormalTok{(ggspatial)}
\CommentTok{\# rosm::osm.types()}
\NormalTok{od\_to\_pubs\_result }\SpecialCharTok{\%\textgreater{}\%} 
  \FunctionTok{ggplot}\NormalTok{() }\SpecialCharTok{+}
  \FunctionTok{annotation\_map\_tile}\NormalTok{(}\AttributeTok{type =} \StringTok{"cartolight"}\NormalTok{) }\SpecialCharTok{+}
  \FunctionTok{geom\_sf}\NormalTok{(}\FunctionTok{aes}\NormalTok{(}\AttributeTok{lwd =}\NormalTok{ interaction, }\AttributeTok{colour =}\NormalTok{ D), }\AttributeTok{alpha =} \FloatTok{0.5}\NormalTok{) }\SpecialCharTok{+}
  \FunctionTok{scale\_size\_continuous}\NormalTok{(}\AttributeTok{range =} \FunctionTok{c}\NormalTok{(}\FloatTok{0.3}\NormalTok{, }\DecValTok{3}\NormalTok{)) }\SpecialCharTok{+}
  \FunctionTok{geom\_sf}\NormalTok{(}\AttributeTok{data =}\NormalTok{ origin\_zones, }\AttributeTok{fill =} \ConstantTok{NA}\NormalTok{, }\AttributeTok{lty =} \DecValTok{2}\NormalTok{, }\AttributeTok{alpha =} \FloatTok{0.5}\NormalTok{) }\SpecialCharTok{+}
  \FunctionTok{theme\_void}\NormalTok{()}
\end{Highlighting}
\end{Shaded}

\begin{figure}
\includegraphics[width=0.8\linewidth]{main_files/figure-latex/pubresmap-1} \caption{Results of a reproducible SIM undertaken on a minimal example based on synthetic data representing hypothetical trips to 2 pubs in Leeds, UK.}\label{fig:pubresmap}
\end{figure}

\hypertarget{capturing-heterogeneity}{%
\subsection{\texorpdfstring{Capturing Heterogeneity }{Capturing Heterogeneity }}\label{capturing-heterogeneity}}

We argue that research on SIMs should seek to adopt and adapt modelling frameworks to capture and understand patterns of spatial, temporal and population non-stationarity which can now be capture given the rise of large data sets.
This line of enquire involves calibrating models to estimate separate parameters for individual origins, destinations, time units and population segments.
Such parameters can reflect local, temporal and population variations in the relationships producing spatial flows.
Inference based on a set of global model parameters may lead to draw misleading conclusions due to misrepresentations of local-, temporal- and population-specific trends.

Yet, as noted above, limited progress has been made on capturing these patterns of non-stationarity (Oshan 2021).
Existing approaches have been proposed to capture spatial-nonstationarity.
An approach involves subsetting the overall data set into origin- or destination-specific data sets and calibrate individual models for each data set.
An alternative approach is to include interaction terms between a binary categorical variable identifying an origin or destination, and each of the relevant covariates in a regression model.
A third alternative is to use geographically weighted regressions (GWRs) to capture spatial non-stationarity (Graells-Garrido et al. 2021), but the extension of these models to calibrate spatial flow count data is challenging.
It often requires the log transformation of spatial flows.
Yet, such approaches may be inappropriate when dealing with sparse origin-destination matrices containing zeros.
In such scenarios, using appropriate count distributions is recommended (O'Hara and Kotze 2010).

We propose generalised linear mixed models (GLMMs) as a more flexible modelling framework to capture all three sources of non-stationarity based on appropriate count data distributions.
GLMMs extend GLMs to incorporate a combination of random and fixed effects parameters as predictor variables, and accommodate non-continuous responses, such as binary and count responses.
Fixed effects represent a typical covariate and are typically used to capture ``global'' average patterns.
Random effects are represented by categorical variables encoding some grouping unit, and can be used to estimate the extent of variations between and within groping units.
Random effects are flexible, and in SIMs, units could comprise groups or individual origins, destinations, origin-destination pairs, time intervals or population subgroups to capture spatial, temporal and population variations in spatial flows.
Unlike GWRs, the flexibility of GLMMs provides an opportunity selectively capture patterns of non-stationarity in relationships with a selected group of theoretically- or policy-relevant variables.
In addition to flexibility, random effects can aid to correct statistical inference about fixed ``global'' effects by providing an estimation of variable in the response variables within and between groups.
They can also reduce the probability of Type I error and Type I error (Harrison et al. 2018).
GLMMs can also be used to explicitly model spatial and temporal auto-correlation.

Yet, challenges exist in applying GLMMs.
First, GLMMs make additional assumptions about the data to those made in standard statistical approaches and they need to be tested {[}{]}.
Second, interpreting the model outputs from GLMMs correctly may be challenging, particularly estimates relating to variance components of random effects and correlations.
Third, model selection is a challenge because of biases in model performance tests caused by the presence of random effects {[}{]}.
Guidelines for the implementation of GLMMs will therefore be needed to navegate these challenges and leverage the potential of these models to effectively account for non-stationarity in spatial flows.

\hypertarget{enhancing-statistical-inference}{%
\subsection{\texorpdfstring{Enhancing Statistical Inference }{Enhancing Statistical Inference }}\label{enhancing-statistical-inference}}

We argue new ways of approaching statistical inference in SIMs.
First, we call for a careful use of the concept of statistical significance.
As highlighted in Section \ref{big-data-modelling}, dichotomising estimates into `statistically significant' and `statistically non-significant' is unhelpful and can lead to draw misleading conclusions, particularly for models relying on large data sets as the full set of estimates in such models can render \emph{P} values below 0.05.
An approach is embracing uncertainty, and re-conceptualing confidence intervals as `compatibility intervals' can provide a practical solution (Amrhein, Greenland, and McShane 2019).
This shifts the focus to all values inside the interval and to the fact that singling out a single point estimate may not be appropriate to draw broad conclusions about the factors underpining spatial flows.

Second, we argue for greater use of causal inferential approaches.
Big Data now offers an unprecedented temporal frequency to capture spatial interactions in very short time frames, and understand the sequence of events to distinguish causes and consequences of these interactions.
Identifying these causal-effect relationships is key to understand the impact of interventions and inform the development of policies aiming at generating a desired outcome.
Inference statistical approaches are widely used in other social science disciplines, but these practices have not permeated through geography.
Causal inference on spatial processes faces additional challenges, such as spatial dependence, spatial heterogeneity and spatial effects (Akbari, Winter, and Tomko 2021).
So, while adopting causal inference methods to analyse spatial interactions may not be straightforward, we believe that is a valuable endeavor to inform the design of future policy interventions.

Third, we propose the use of multi-model inference.
Intuitively, multi-model inference seeks to draw conclusions from a range of theoretically-sound model specifications (Burnham and Anderson 2004).
It uses model averaging to determine the direction and strength of regression predictors, and generating ways assess their relative importance based on Akaike Information Criterion scores from multiple models.
As such, inferences are not drawn from a single model, leading to more robust inferences.
Determining the relevance of factors shaping spatial flows based on multi-model inference offers an additional alternative to make inference in the context of SIMs and Big Data.
Multi-model inference has been rarely applied to capture spatial interactions (Rowe 2013).
Yet, it is widely used in biology and ecology, and various routines exist in \emph{R} which can be integrated with \texttt{simodels}.

\hypertarget{integrating-data-science}{%
\subsection{\texorpdfstring{Integrating Data Science }{Integrating Data Science }}\label{integrating-data-science}}

New methods and tools are needed to engineer and calibrate SIMs using large data sets.
In agreement with Singleton and Arribas-Bel (2019), we recognise that the value of data science approaches to enable the calibration of SIMs on large data sets.
We propose that data science can be beneficial on three key fronts.
First, a key challenge is the storage, manipulation and analysis of large volumes of data.
Large data sets cannot generally be stored and analysed on a computer's local memory.
Traditional approaches relying on local memory capacity to calibrate SIMs are therefore sub-optimal.
Adoption of data science approaches to Big Data storage, scalability, high performance computing and parallel computing.
Training and infrastructure to integrate these approaches are needed to enable researchers to unlock the opportunities afforded by Big Data, understanding spatial interactions in real-time or near real-time at more detailed temporal and spatial granularity.

Second, we propose the use of ML/AI to enhance the prediction capacity of SIMs.
While this proposal is not new and ML/AI approaches have been deployed mainly in transport applications, their deployment in geography has been limited.
Big Data now provides enough density of data to train and assess the capability of these algorithms to predict spatial interactions.
Emerging evidence suggests that ML-calibrated SIMs outperform traditional GLM-calibrated SIMs at generated more accurate predictions of spatial interactions (Rowe, Mahony, and Tao 2022).
ML/AI models do not require a pre-define functional specification.
They have the ability to uncover complex structure in the data and are rapidly deployable requiring limited human intervention.
In the context of SIMs, the promise is that ML/AI can uncover and accommodate complex functional forms capturing patterns of spatial and temporal dependence and non-stationarity (Rowe, Mahony, and Tao 2022).

\begin{itemize}
\tightlist
\item
  Causal inference
\end{itemize}

use of machine learning for prediction and inference -

scalable models

Remaining challenges - reproducibility (technical enabling infrastructure), calibration and leverage on the potential opportunities offered by big data - traditional and new forms of data - leverage on properties -- real-time information, large volumes and temporal frequency - machine learning to improve inference and prediction as well as address issues of scalability (suggest broad approaches: parallel computing, slicing the data), inference (discuss how p-value is not a useful indicator anymore) and capturing heterogeneity.

\hypertarget{the-future-of-spatial-interaction-modelling}{%
\section{The future of spatial interaction modelling}\label{the-future-of-spatial-interaction-modelling}}

Objective of the section: To identify and discuss the key pillars that will enable progress on all the proposed fronts - open science, important and new questions and digital infrastructure.
I see this as our conclusion - probably one or two short paragraph summarising what has been discussed with a forward looking approach.

\begin{quote}
\end{quote}

\hypertarget{references}{%
\section*{References}\label{references}}
\addcontentsline{toc}{section}{References}

\hypertarget{refs}{}
\begin{CSLReferences}{1}{0}
\leavevmode\vadjust pre{\hypertarget{ref-akbari2021}{}}%
Akbari, Kamal, Stephan Winter, and Martin Tomko. 2021. {``Spatial Causality: A Systematic Review on Spatial Causal Inference.''} \emph{Geographical Analysis}, December. \url{https://doi.org/10.1111/gean.12312}.

\leavevmode\vadjust pre{\hypertarget{ref-amrhein2019}{}}%
Amrhein, Valentin, Sander Greenland, and Blake McShane. 2019. {``Scientists Rise up Against Statistical Significance.''} \emph{Nature} 567 (7748): 305--7. \url{https://doi.org/10.1038/d41586-019-00857-9}.

\leavevmode\vadjust pre{\hypertarget{ref-arribas-bel2021}{}}%
Arribas-Bel, Dani, Mark Green, Francisco Rowe, and Alex Singleton. 2021. {``Open Data Products-A Framework for Creating Valuable Analysis Ready Data.''} \emph{Journal of Geographical Systems} 23 (4): 497--514. \url{https://doi.org/10.1007/s10109-021-00363-5}.

\leavevmode\vadjust pre{\hypertarget{ref-brunsdon2016}{}}%
Brunsdon, Chris. 2016. {``Quantitative Methods I.''} \emph{Progress in Human Geography} 40 (5): 687--96. \url{https://doi.org/10.1177/0309132515599625}.

\leavevmode\vadjust pre{\hypertarget{ref-modelse2004}{}}%
Burnham, Kenneth P., and David R. Anderson, eds. 2004. \emph{Model Selection and Multimodel Inference}. Springer New York. \url{https://doi.org/10.1007/b97636}.

\leavevmode\vadjust pre{\hypertarget{ref-fotheringham1983}{}}%
Fotheringham, A S. 1983. {``A New Set of Spatial-Interaction Models: The Theory of Competing Destinations.''} \emph{Environment and Planning A: Economy and Space} 15 (1): 15--36. \url{https://doi.org/10.1177/0308518x8301500103}.

\leavevmode\vadjust pre{\hypertarget{ref-graells-garrido2021}{}}%
Graells-Garrido, Eduardo, Feliu Serra-Burriel, Francisco Rowe, Fernando M. Cucchietti, and Patricio Reyes. 2021. {``A City of Cities: Measuring How 15-Minutes Urban Accessibility Shapes Human Mobility in Barcelona.''} Edited by Wenjia Zhang. \emph{PLOS ONE} 16 (5): e0250080. \url{https://doi.org/10.1371/journal.pone.0250080}.

\leavevmode\vadjust pre{\hypertarget{ref-harrison2018}{}}%
Harrison, Xavier A., Lynda Donaldson, Maria Eugenia Correa-Cano, Julian Evans, David N. Fisher, Cecily E. D. Goodwin, Beth S. Robinson, David J. Hodgson, and Richard Inger. 2018. {``A Brief Introduction to Mixed Effects Modelling and Multi-Model Inference in Ecology.''} \emph{PeerJ} 6 (May): e4794. \url{https://doi.org/10.7717/peerj.4794}.

\leavevmode\vadjust pre{\hypertarget{ref-hunt2004}{}}%
Hunt, Len M., Barry Boots, and Pavios S. Kanaroglou. 2004. {``Spatial Choice Modelling: New Opportunities to Incorporate Space into Substitution Patterns.''} \emph{Progress in Human Geography} 28 (6): 746--66. \url{https://doi.org/10.1191/0309132504ph517oa}.

\leavevmode\vadjust pre{\hypertarget{ref-mcfadden1974analysis}{}}%
McFadden, D. 1974. \emph{Analysis of Qualitative Choice Behavior. Zarembka, p.(ed.): Frontiers in Econometrics}. Academic Press. New York, NY.

\leavevmode\vadjust pre{\hypertarget{ref-niedomysl2017}{}}%
Niedomysl, Thomas, Ola Hall, Maria Francisca Archila Bustos, and Ulf Ernstson. 2017. {``Using Satellite Data on Nighttime Lights Intensity to Estimate Contemporary Human Migration Distances.''} \emph{Annals of the American Association of Geographers} 107 (3): 591--605. \url{https://doi.org/10.1080/24694452.2016.1270191}.

\leavevmode\vadjust pre{\hypertarget{ref-nust_practical_2021}{}}%
Nüst, Daniel, and Edzer Pebesma. 2021. {``Practical Reproducibility in Geography and Geosciences.''} \emph{Annals of the American Association of Geographers} 111 (5): 1300--1310. \url{https://doi.org/10.1080/24694452.2020.1806028}.

\leavevmode\vadjust pre{\hypertarget{ref-ohara2010}{}}%
O'Hara, Robert, and Johan Kotze. 2010. {``Do Not Log-Transform Count Data.''} \emph{Nature Precedings}, January. \url{https://doi.org/10.1038/npre.2010.4136.1}.

\leavevmode\vadjust pre{\hypertarget{ref-ogcopengeospatialconsortiuminc_opengis_2011}{}}%
(OGC) Open Geospatial Consortium Inc. 2011. {``OpenGIS Implementation Specification for Geographic Information - Simple Feature Access - Part 1: Common Architecture.''} \url{https://www.ogc.org/standards/sfa}.

\leavevmode\vadjust pre{\hypertarget{ref-oshan2021spatial}{}}%
Oshan, Taylor M. 2021. {``The Spatial Structure Debate in Spatial Interaction Modeling: 50 Years On.''} \emph{Progress in Human Geography} 45 (5): 925--50.

\leavevmode\vadjust pre{\hypertarget{ref-pappalardo2015}{}}%
Pappalardo, Luca, Filippo Simini, Salvatore Rinzivillo, Dino Pedreschi, Fosca Giannotti, and Albert-László Barabási. 2015. {``Returners and Explorers Dichotomy in Human Mobility.''} \emph{Nature Communications} 6 (1). \url{https://doi.org/10.1038/ncomms9166}.

\leavevmode\vadjust pre{\hypertarget{ref-popper_logic_1934}{}}%
Popper, Karl. 1934. \emph{The Logic of Scientific Discovery}. Hutchinson. \url{http://books.google.com/books?id=MdvaSAAACAAJ\&pgis=1}.

\leavevmode\vadjust pre{\hypertarget{ref-rowe2013}{}}%
Rowe, Francisco. 2013. {``Spatial Labour Mobility in a Transition Economy: Migration and Commuting in Chile.''} PhD thesis. \url{https://doi.org/10.14264/uql.2017.427}.

\leavevmode\vadjust pre{\hypertarget{ref-rowe2021}{}}%
---------. 2021. {``Big Data and Human Geography.''} \url{http://dx.doi.org/10.31235/osf.io/phz3e}.

\leavevmode\vadjust pre{\hypertarget{ref-rowe2022}{}}%
Rowe, Francisco, Michael Mahony, and Sui Tao. 2022. {``Assessing Machine Learning Algorithms for Near-Real Time Bus Ridership Prediction During Extreme Weather.''} \url{https://doi.org/10.48550/ARXIV.2204.09792}.

\leavevmode\vadjust pre{\hypertarget{ref-rowe2020}{}}%
Rowe, Francisco, Gunther Maier, Daniel Arribas-Bel, and Sergio Rey. 2020. {``The Potential of Notebooks for Scientific Publication, Reproducibility and Dissemination.''} \emph{REGION} 7 (3): E1--5. \url{https://doi.org/10.18335/region.v7i3.357}.

\leavevmode\vadjust pre{\hypertarget{ref-schwartz1973interpreting}{}}%
Schwartz, Aba. 1973. {``Interpreting the Effect of Distance on Migration.''} \emph{Journal of Political Economy} 81 (5): 1153--69.

\leavevmode\vadjust pre{\hypertarget{ref-simini_universal_2012}{}}%
Simini, Filippo, Marta C González, Amos Maritan, and Albert-László Barabási. 2012. {``A Universal Model for Mobility and Migration Patterns.''} \emph{Nature}, February, 812. \url{https://doi.org/10.1038/nature10856}.

\leavevmode\vadjust pre{\hypertarget{ref-singleton2019}{}}%
Singleton, Alex, and Daniel Arribas-Bel. 2019. {``Geographic Data Science.''} \emph{Geographical Analysis} 53 (1): 61--75. \url{https://doi.org/10.1111/gean.12194}.

\leavevmode\vadjust pre{\hypertarget{ref-wilson_family_1971}{}}%
Wilson, AG. 1971. {``A Family of Spatial Interaction Models, and Associated Developments.''} \emph{Environment and Planning} 3 (January): 132. https://doi.org/\url{https://doi.org/10.1068/a030001}.

\end{CSLReferences}




% print the bibliography
\setlength{\bibsep}{0.00cm plus 0.05cm} % no space between items
\bibliographystyle{apalike}
\bibliography{sim_refs}



\end{document}
