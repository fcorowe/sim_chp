\RequirePackage[l2tabu,orthodox]{nag}
\documentclass[11pt,letterpaper]{article}
\usepackage[T1]{fontenc}
\usepackage[utf8]{inputenc}
\usepackage{crimson}
\usepackage{helvet}
\usepackage[strict,autostyle]{csquotes}
\usepackage[USenglish]{babel}
\usepackage{microtype}
\usepackage{authblk}
\usepackage{booktabs}
\usepackage{caption}
\usepackage{endnotes}
\usepackage{geometry}
\usepackage{graphicx}
\usepackage{hyperref}
\usepackage{natbib}
\usepackage{rotating}
\usepackage{setspace}
\usepackage{titlesec}
\usepackage{url}
\usepackage{soul}
\usepackage[dvipsnames]{xcolor}
\usepackage[many]{tcolorbox}
\newtcolorbox{mybox}{colback = black!5!gray!50, colframe = black!75!black, segmentation style={solid}} % create text box
\usepackage{hanging}

% location of figure files, via graphicx package
\graphicspath{{./figures/}}

% configure the page layout, via geometry package
\geometry{
	paper=letterpaper,
	top=4cm,
	bottom=4cm,
	left=4cm,
	right=4cm}
\setstretch{1.02}
\clubpenalty=10000
\widowpenalty=10000

% set section/subsection headings as the sans serif font
\titleformat{\section}{\normalfont\sffamily\large\bfseries}{\thesection.}{0.3em}{}
\titleformat{\subsection}{\normalfont\sffamily\small\bfseries}{\thesubsection.}{0.3em}{}

% make figure/table captions sans-serif small font
\captionsetup{font={footnotesize,sf},labelfont=bf,labelsep=period}

% configure pdf metadata and link handling
\hypersetup{
	pdfauthor={Francisco Rowe, Robin Lovelace, Adam Dennett},
	pdftitle={Gravity and Spatial Interaction Modelling: A Manisfesto},
	pdfsubject={Gravity and Spatial Interaction Modelling: A Manisfesto},
	pdfkeywords={Spatial interaction modelling, Gravity modelling, Machine learning, Geographic data science},
	pdffitwindow=true,
	breaklinks=true,
	colorlinks=false,
	pdfborder={0 0 0}}

\title{Gravity and Spatial Interaction Modelling: A Manisfesto\footnote{\textbf{Citation}: Rowe, F. Lovelace, R. Dennett, A., 2022. Gravity and Spatial Interaction Modelling: A Manisfesto. In: Wolf, L., Heppenstall, A., and Harris, R. (eds) Edward Elgar Books}}
\author[1]{Francisco Rowe \thanks{\textit{Email}: F.Rowe-Gonzalez@liverpool.ac.uk}}
\affil[1]{Geographic Data Science Lab, Department of Geography and Planning, University of Liverpool, Liverpool, United Kingdom}
\author[2]{Robin Lovelace \thanks{\textit{Email}: R.Lovelace@leeds.ac.uk}}
\affil[2]{Institute for Transport Studies, University of Leeds, Leeds, United Kingdom}
\author[3]{Adam Dennett \thanks{\textit{Email}: a.dennet@ucl.ac.uk}}
\affil[3]{The Bartlett Centre for Advanced Spatial Analytics, University College London, London, United Kingdom}

\date{}

% From pandoc:
% https://github.com/jgm/pandoc-templates/blob/master/default.latex
\newlength{\cslhangindent}
\setlength{\cslhangindent}{1.5em}
\newlength{\csllabelwidth}
\setlength{\csllabelwidth}{3em}
\newlength{\cslentryspacingunit} % times entry-spacing
\setlength{\cslentryspacingunit}{\parskip}
\newenvironment{CSLReferences}[2] % #1 hanging-ident, #2 entry spacing
 {% don't indent paragraphs
  \setlength{\parindent}{0pt}
  % turn on hanging indent if param 1 is 1
  \ifodd #1
  \let\oldpar\par
  \def\par{\hangindent=\cslhangindent\oldpar}
  \fi
  % set entry spacing
  \setlength{\parskip}{#2\cslentryspacingunit}
 }%
 {}
\usepackage{calc}
\newcommand{\CSLBlock}[1]{#1\hfill\break}
\newcommand{\CSLLeftMargin}[1]{\parbox[t]{\csllabelwidth}{#1}}
\newcommand{\CSLRightInline}[1]{\parbox[t]{\linewidth - \csllabelwidth}{#1}\break}
\newcommand{\CSLIndent}[1]{\hspace{\cslhangindent}#1}
\setlength{\emergencystretch}{3em} % prevent overfull lines
\providecommand{\tightlist}{%
  \setlength{\itemsep}{0pt}\setlength{\parskip}{0pt}}

% https://stackoverflow.com/questions/41052687/rstudio-pdf-knit-fails-with-environment-shaded-undefined-error
\usepackage{color}
\usepackage{fancyvrb}
\newcommand{\VerbBar}{|}
\newcommand{\VERB}{\Verb[commandchars=\\\{\}]}
\DefineVerbatimEnvironment{Highlighting}{Verbatim}{commandchars=\\\{\}}
% Add ',fontsize=\small' for more characters per line
\usepackage{framed}
\definecolor{shadecolor}{RGB}{248,248,248}
\newenvironment{Shaded}{\begin{snugshade}}{\end{snugshade}}
\newcommand{\AlertTok}[1]{\textcolor[rgb]{0.94,0.16,0.16}{#1}}
\newcommand{\AnnotationTok}[1]{\textcolor[rgb]{0.56,0.35,0.01}{\textbf{\textit{#1}}}}
\newcommand{\AttributeTok}[1]{\textcolor[rgb]{0.77,0.63,0.00}{#1}}
\newcommand{\BaseNTok}[1]{\textcolor[rgb]{0.00,0.00,0.81}{#1}}
\newcommand{\BuiltInTok}[1]{#1}
\newcommand{\CharTok}[1]{\textcolor[rgb]{0.31,0.60,0.02}{#1}}
\newcommand{\CommentTok}[1]{\textcolor[rgb]{0.56,0.35,0.01}{\textit{#1}}}
\newcommand{\CommentVarTok}[1]{\textcolor[rgb]{0.56,0.35,0.01}{\textbf{\textit{#1}}}}
\newcommand{\ConstantTok}[1]{\textcolor[rgb]{0.00,0.00,0.00}{#1}}
\newcommand{\ControlFlowTok}[1]{\textcolor[rgb]{0.13,0.29,0.53}{\textbf{#1}}}
\newcommand{\DataTypeTok}[1]{\textcolor[rgb]{0.13,0.29,0.53}{#1}}
\newcommand{\DecValTok}[1]{\textcolor[rgb]{0.00,0.00,0.81}{#1}}
\newcommand{\DocumentationTok}[1]{\textcolor[rgb]{0.56,0.35,0.01}{\textbf{\textit{#1}}}}
\newcommand{\ErrorTok}[1]{\textcolor[rgb]{0.64,0.00,0.00}{\textbf{#1}}}
\newcommand{\ExtensionTok}[1]{#1}
\newcommand{\FloatTok}[1]{\textcolor[rgb]{0.00,0.00,0.81}{#1}}
\newcommand{\FunctionTok}[1]{\textcolor[rgb]{0.00,0.00,0.00}{#1}}
\newcommand{\ImportTok}[1]{#1}
\newcommand{\InformationTok}[1]{\textcolor[rgb]{0.56,0.35,0.01}{\textbf{\textit{#1}}}}
\newcommand{\KeywordTok}[1]{\textcolor[rgb]{0.13,0.29,0.53}{\textbf{#1}}}
\newcommand{\NormalTok}[1]{#1}
\newcommand{\OperatorTok}[1]{\textcolor[rgb]{0.81,0.36,0.00}{\textbf{#1}}}
\newcommand{\OtherTok}[1]{\textcolor[rgb]{0.56,0.35,0.01}{#1}}
\newcommand{\PreprocessorTok}[1]{\textcolor[rgb]{0.56,0.35,0.01}{\textit{#1}}}
\newcommand{\RegionMarkerTok}[1]{#1}
\newcommand{\SpecialCharTok}[1]{\textcolor[rgb]{0.00,0.00,0.00}{#1}}
\newcommand{\SpecialStringTok}[1]{\textcolor[rgb]{0.31,0.60,0.02}{#1}}
\newcommand{\StringTok}[1]{\textcolor[rgb]{0.31,0.60,0.02}{#1}}
\newcommand{\VariableTok}[1]{\textcolor[rgb]{0.00,0.00,0.00}{#1}}
\newcommand{\VerbatimStringTok}[1]{\textcolor[rgb]{0.31,0.60,0.02}{#1}}
\newcommand{\WarningTok}[1]{\textcolor[rgb]{0.56,0.35,0.01}{\textbf{\textit{#1}}}}


\begin{document}

\maketitle


\textbf{NOTE} This is a preprint version of the chapter. Please do not redistribute without
permission of the authors. You can get in touch with Francisco Rowe at
F.Rowe-Gonzalez@liverpool.ac.uk; Robin Lovelace at R.Lovelace@leeds.ac.uk or Adam Dennett a.dennet@ucl.ac.uk

\begin{abstract}


%\vspace{1cm}
\end{abstract}



\pagebreak

Overall aim: To set our view about how the field of spatial interaction modelling should evolve.

Key argument: Spatial interaction models are great, yet, not progress has been made over the last two/three decades (though we should recognise the work LeSage on spatial econometrics, Dan Griffith on spatial filtering, Kanaroglou et al on spatial discrete choice modelling, radiation models).
Key challenges have prevented progress:

1.
reproducibility

2.
calibration

3.
large volumes of granular spatial data .

Context: spatial interaction modelling in a world of big data, machine learning, open science, digital technology and uncertainty.

The text included here is for our own benefit.
The idea of challenges

\hypertarget{definition-and-uses}{%
\section{Definition and Uses}\label{definition-and-uses}}

\begin{itemize}
\item
  What are spatial interaction models?
  Spatial interaction models (SIMs) definition - first intuitive definition / explanation - conceptually describe what spatial interaction modelling is (including its key components - talk about unconstrained and constrained models. But I don't think we need to spend tons of words on this.
\item
  Why are SIMs important?
  Applications and context - what they allow us to do: retail (delineating market areas), migration, human mobility, transport (factors influencing the size of people travelling), urban planning (), what if scenarios (forecasting)
\item
  Progress on SIMs - recognise the work LeSage on spatial econometrics, Dan Griffith on spatial filtering, Kanaroglou et al on spatial discrete choice modelling, radiation models
\item
  Remaining challenges - reproducibility (technical infrastructure)
\end{itemize}

Objective of the section: To conceptually describe what spatial interaction modelling is (including its key components), how it is used, why it is important and how it relates to gravity modelling.
I conceive this section to provide a brief, gentle, intuitive introduction to spatial interaction models, emphasising its importance and the various context of applications of SIMs i.e.~retail, population, transport, etc.

\begin{itemize}
\tightlist
\item
  Mention of key SIM paradigms
\end{itemize}

SIMs cover a wide range of methods and applications.
According to Rodrigue, Comtois, and Slack (2013), there are 3 broad types of SIM:

\begin{itemize}
\item
  Gravity models, in which interaction is estimated as a function of size/attractiveness of start/end points and some impedance function; this is the original and `traditional' SIM.
\item
  Radiation models or `potential models', in which interaction is estimated as a function of size/attractiveness of start/end points but mediated by a function of intervening opportunities (Simini et al. 2012).
\item
  Retail models, which seek to identify the `market boundary' between economic hubs.
\end{itemize}

For the majority of this paper we will focus on gravity models.
There are four main types of traditional SIMs (Wilson 1971) :

\begin{itemize}
\item
  Unconstrained
\item
  Production-constrained
\item
  Attraction-constrained
\item
  Doubly-constrained
\end{itemize}

The basic unconstrained SIM can be defined was follows, in a paper that explored many iterations on this formulation:

\[
T_{i j}=K \frac{W_{i}^{(1)} W_{j}^{(2)}}{c_{i j}^{n}}
\] ``where \(T_{i j}\) is a measure of the interaction between zones \(i\) and \(W_{i}^{(1)}\) is a measure of the `mass term' associated with zone \(z_i\), \(W_{j}^{(2)}\) is a measure of the `mass term' associated with zone \(z_j\), and \(c_{ij}\) is a measure of the distance, or generalised cost of travel, between zone \(i\) and zone \(j\)''.
\(K\) is a `constant of proportionality' and \(n\) is a parameter to be estimated.

Redefining the \(W\) terms as \(m\) and \(n\) for origins and destinations respectively (Simini et al. 2012), this classic definition of the `gravity model' can be written as follows:

\[
T_{i j}=K \frac{m_{i} n_{j}}{c_{i j}^{n}}
\]

Identify reproducibility, calibration and big data as key challenges as noted above.
If we frame these as key challenges, I think it would make sense to map the rest of the sections onto these challenges.
The issue I have on doing this is that the issues of calibration and big data may entail similar or the same challenges.

\hypertarget{expanding-spatial-interaction-modelling}{%
\section{Expanding spatial interaction modelling}\label{expanding-spatial-interaction-modelling}}

Objective of the section: To discuss ways in which spatial interaction modelling can be expanded.
I am particularly interested in advocating for (1) seeing spatial interaction models in a context of uncertainty, rather than as mathematical models; and, (2) discussing how hierarchical / generalised linear mixed modelling can offer ways to capture variability across places and populations.

\hypertarget{reproducible-sims}{%
\section{\texorpdfstring{Reproducible SIMs }{Reproducible SIMs }}\label{reproducible-sims}}

\emph{Objective of the section: To discuss the opportunities and challenges of estimating spatial interacting modelling in the context of reproducible research}

Reproducibility has not been prominent in research developing and using SIMs outlined in the previous sections.
This is understandable, because computer hardware, software and know-how needed to develop and run SIMs was simply unavailable to most people (let alone lowly and often cash-strapped students!) for most of field's history.
Even when consumer laptops became widely available and more affordable during the 2000s, there were few well-known user-friendly off-the-shelf options implementing SIMs other than MATLAB and arguably Excel, unless you were willing to dive into programming.

Fast-forward to the 2020s and computer hardware, and perhaps more importantly software, is much easier to obtain.
In terms of affordability, a second-hand laptop whose processing power would have been considered a supercomputer by 1990s standards (and impossibly powerful when seminal SIM papers were published) can be obtained for around \$100 dollars in most countries.\footnote{See \url{https://ebay.com/b/Laptops-Netbooks/175672/bn_1648276} for an example of the huge international market for second-hand laptops.}
Tutorials teaching you to code with popular languages for data science abound, with R and Python particularly prominent in the fields (including Quantitative Geography, Social Physics, Statistics and more recently Urban Analytics) where much SIM research is undertaken.
It has never been easier to write reproducible code implementing SIMs yet, despite notable exceptions (Dennett 2018), most SIMs and the findings they produce, are not reproducible.

One could argue that reproducibility is a `nice to have', an optional and potentially onerous extra thing to think about during the research process.
Yet increasingly it is becoming apparent that reproducibility is \emph{vital} for research to be falsifiable (Popper 1934) and therefore scientifically sound.
In this broader context, reproducibility is a ``challenge to adjust scholarly communication to today's level of digitisation and diversity of scientific outputs'' (Nüst and Pebesma 2021).
A key message in our manifesto for SIM research in the 2020s, therefore, is for the models to be open and the results they create to be reproducible using published code and data (example synthetic data when the raw data cannot be published).
We urge readers implementing SIMs to make their work reproducible not only for philosophical reasons.
There are tangible benefits of making your SIM work reproducible:

\begin{itemize}
\item
  People are more likely to cite your work if they can reproduce it.
\item
  Reproducible results based on open source software discourages reinvention of wheels and associated wasting of time.
\item
  Reproducible research encourages innovation, both of your work and the work of others, because you can focus on what is new and novel rather than (for example) writing a paper implementing an existing SIM in a slightly new context (as many SIM papers have).
\end{itemize}

To highlight the ease with which reproducible SIMs can now be developed, we present below reproducible R code that implements a simple SIM.
It is notable that open source software continues to evolve: the code presented below uses the \texttt{simodels} R package, the development of which was partly motivated by this book chapter (the primary motivation was the need to develop SIMs to represent trips for purposes other than commuting and travel to school in Ireland as part of a contract with Transport Infrastructure Ireland, highlighting the applied nature of SIM research).
\texttt{simodels} enables to develop SIMs starting with geographic datasets in fewer lines of code than was possible a few years ago (Dennett 2018).
Naturally, the starting point is to install the package (and R and a modern IDE such as RStudio or \href{https://marketplace.visualstudio.com/items?itemName=REditorSupport.r}{VS Code} if the software is not already installed on your computer), with the following lines of code:

\begin{Shaded}
\begin{Highlighting}[]
\FunctionTok{install.packages}\NormalTok{(}\StringTok{"simodels"}\NormalTok{)}
\end{Highlighting}
\end{Shaded}

The package installed with the previous command, which is called \texttt{simodels} (short for spatial interaction models) does not just provide functions for running and fitting (finding parameters to minimise model-observation differences): it provides a framework for developing SIMs and creating new functions implementing different types of SIM and using a variety of pre-existing modelling tools in SIMs.
We will also install \texttt{tidyverse} (if not already) for intuitive data processing functionality, and load (technically attach) the packages so their functions are available:

\begin{Shaded}
\begin{Highlighting}[]
\FunctionTok{install.packages}\NormalTok{(}\StringTok{"tidyverse"}\NormalTok{)}
\end{Highlighting}
\end{Shaded}

\begin{Shaded}
\begin{Highlighting}[]
\FunctionTok{library}\NormalTok{(simodels)}
\FunctionTok{library}\NormalTok{(tidyverse)}
\end{Highlighting}
\end{Shaded}

The starting `point' (pun intended!) of all SIMs is geographic entities representing trip start, end or (for `multi-partite' models) or intermediate points.
We use the word `features' because almost all SIMs use input datsets that are compliant with the `simple features' open specification ((OGC) Open Geospatial Consortium Inc 2011), typically imported from files encoded in proprietary the Shapefile (\texttt{.shp}) or open GeoPackage (\texttt{.gpkg}), GeoJSON (\texttt{.geojson}) or other geographic file formats.
R has a mature ecosystem for working with geographic file formats, so we can use existing function for this data import stage, using the \texttt{sf} package:

\begin{Shaded}
\begin{Highlighting}[]
\NormalTok{u\_origins }\OtherTok{=} \StringTok{"origin\_zones.geojson"}
\NormalTok{f\_origins }\OtherTok{=} \FunctionTok{basename}\NormalTok{(u\_origins)}
\NormalTok{u\_destinations }\OtherTok{=} \StringTok{"destination\_points.geojson"}
\NormalTok{f\_destinations }\OtherTok{=} \FunctionTok{basename}\NormalTok{(u\_destinations)}
\end{Highlighting}
\end{Shaded}

\begin{Shaded}
\begin{Highlighting}[]
\FunctionTok{download.file}\NormalTok{(u\_origins, }\AttributeTok{destfile =}\NormalTok{ f\_origins)}
\FunctionTok{download.file}\NormalTok{(u\_destinations, }\AttributeTok{destfile =}\NormalTok{ f\_destinations)}
\end{Highlighting}
\end{Shaded}

\begin{Shaded}
\begin{Highlighting}[]
\NormalTok{origin\_zones }\OtherTok{=}\NormalTok{ sf}\SpecialCharTok{::}\FunctionTok{read\_sf}\NormalTok{(}\StringTok{"origin\_zones.geojson"}\NormalTok{)}
\NormalTok{destination\_points }\OtherTok{=}\NormalTok{ sf}\SpecialCharTok{::}\FunctionTok{read\_sf}\NormalTok{(}\StringTok{"destination\_points.geojson"}\NormalTok{)}
\end{Highlighting}
\end{Shaded}

The code chunk above demonstrates importing specific data objects:

\begin{itemize}
\item
  A simple features object with `multipolygon' geometries representing administrative zones that constitute trip origins in the subsequent reproducible SIMs.
\item
  Another simple features object with `point' geometries representing two popular pubs in Leeds that are trip destinations in the SIMs below.
\item
\item
  patial interaction modelling using machine learning
\end{itemize}

Before creating SIMs representing travel to these two pubs in Leeds, a deliberately minimal and simple input dataset to aid understanding, it is worth doing a small amount of exploratory data analysis (EDA) to check the input datasets.
This is undertaken with the following commands:

\begin{Shaded}
\begin{Highlighting}[]
\FunctionTok{hist}\NormalTok{(origin\_zones}\SpecialCharTok{$}\NormalTok{all)}
\end{Highlighting}
\end{Shaded}

\includegraphics{main_files/figure-latex/unnamed-chunk-7-1.pdf}

\begin{Shaded}
\begin{Highlighting}[]
\FunctionTok{plot}\NormalTok{(sf}\SpecialCharTok{::}\FunctionTok{st\_geometry}\NormalTok{(origin\_zones))}
\FunctionTok{plot}\NormalTok{(sf}\SpecialCharTok{::}\FunctionTok{st\_geometry}\NormalTok{(destination\_points), }\AttributeTok{add =} \ConstantTok{TRUE}\NormalTok{)}
\end{Highlighting}
\end{Shaded}

\includegraphics{main_files/figure-latex/unnamed-chunk-7-2.pdf}

\hypertarget{spatial-interaction-modelling-using-machine-learning}{%
\section{Spatial interaction modelling using machine learning}\label{spatial-interaction-modelling-using-machine-learning}}

Objective of the section: To discuss how machine learning can enhance flow count inference and prediction

\hypertarget{facilitating-future-progress-of-spatial-interaction-modelling}{%
\section{Facilitating future progress of spatial interaction modelling}\label{facilitating-future-progress-of-spatial-interaction-modelling}}

Objective of the section: To identify and discuss the key pillars that will enable progress on all the proposed fronts - open science, important and new questions and digital infrastructure.
I see this as our conclusion - probably one or two short paragraph summarising what has been discussed with a forward looking approach.

\hypertarget{references}{%
\section*{References}\label{references}}
\addcontentsline{toc}{section}{References}

\hypertarget{refs}{}
\begin{CSLReferences}{1}{0}
\leavevmode\vadjust pre{\hypertarget{ref-dennett_modelling_2018}{}}%
Dennett, Adam. 2018. {``Modelling Population Flows Using Spatial Interaction Models.''} \emph{Australian Population Studies} 2 (2): 33--58. \url{https://doi.org/10.37970/aps.v2i2.38}.

\leavevmode\vadjust pre{\hypertarget{ref-nust_practical_2021}{}}%
Nüst, Daniel, and Edzer Pebesma. 2021. {``Practical Reproducibility in Geography and Geosciences.''} \emph{Annals of the American Association of Geographers} 111 (5): 1300--1310. \url{https://doi.org/10.1080/24694452.2020.1806028}.

\leavevmode\vadjust pre{\hypertarget{ref-ogcopengeospatialconsortiuminc_opengis_2011}{}}%
(OGC) Open Geospatial Consortium Inc. 2011. {``OpenGIS Implementation Specification for Geographic Information - Simple Feature Access - Part 1: Common Architecture.''} \url{https://www.ogc.org/standards/sfa}.

\leavevmode\vadjust pre{\hypertarget{ref-popper_logic_1934}{}}%
Popper, Karl. 1934. \emph{The Logic of Scientific Discovery}. Hutchinson. \url{http://books.google.com/books?id=MdvaSAAACAAJ\&pgis=1}.

\leavevmode\vadjust pre{\hypertarget{ref-rodrigue_geography_2013}{}}%
Rodrigue, Jean-Paul, Claude Comtois, and Brian Slack. 2013. \emph{The Geography of Transport Systems}. 3 edition. London ; New York: Routledge.

\leavevmode\vadjust pre{\hypertarget{ref-simini_universal_2012}{}}%
Simini, Filippo, Marta C González, Amos Maritan, and Albert-László Barabási. 2012. {``A Universal Model for Mobility and Migration Patterns.''} \emph{Nature}, February, 812. \url{https://doi.org/10.1038/nature10856}.

\leavevmode\vadjust pre{\hypertarget{ref-wilson_family_1971}{}}%
Wilson, AG. 1971. {``A Family of Spatial Interaction Models, and Associated Developments.''} \emph{Environment and Planning} 3 (January): 132. https://doi.org/\url{https://doi.org/10.1068/a030001}.

\end{CSLReferences}




% print the bibliography
\setlength{\bibsep}{0.00cm plus 0.05cm} % no space between items
\bibliographystyle{apalike}
\bibliography{sim_refs}



\end{document}
